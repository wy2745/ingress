\section{Conclusion and Future Work}
\label{sec:conclusion}

\name{} is an optimized full \gpu{} virtualization solution, based on the Xen hypervisor, with the adaptive hybrid page table shadowing scheme, which improves performance for workloads with the Massive Update Issue when compared to \gvirt{}. To address this issue, this paper provides a hybrid page table shadowing scheme, i.e., strict and relaxed page table shadowing, to provide an optimized full GPU virtualization based on Xen hypervisor for Intel GPUs. \name{} combines these two page table shadowing mechanisms to reduce VM-exits to the hypervisor. Further, \name{} automatically switches page table between them by detecting \gpu{}'s current workloads, potentially showing significantly improvement to \gvirt{}'s performance for workloads with the Massive Update Issue. In order to decide what type of the page need to be reconstructed, four reconstruction policies are introduced. By running the same testcase through the four policies, the dynamic segmented partial reconstruction policy performs the best.

For future work, we will adapt \name{} to support KVM~{\cite{kivity2007kvm}} when \gvirt{} for KVM is ready. Additionally,  \name{} will be released in the open source community soon. We will focus on the areas of portability, scalability, and scheduling issues. With previous \gpu{} command scheduling methods, such as VGRIS and Pegasus~{\cite{deelman2005pegasus}}, we will investigate the low level access pattern of massive page table modification with the detailed analysis of the performance bottleneck of high level applications. We hope this optimized full \gpu{} virtualization solution gives insight into designing the support of efficient distributed systems for \gpu{} acceleration applications.
