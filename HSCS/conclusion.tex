\section{Conclusion and Future Work}
\label{sec:conclusion}

With the widespread of internet and connected devices, web services are growing in an explosive scale. For users who deploy their web services into kubernetes built in bare machine or private cloud environments,the default ingress controller is not capable of adjusting its load balancing strategy with the demands of web services. That is, the great difference between resource consumption of APIs and the dependency between services require a more fine-grained load balancing algorithm and the scaling of web services requires load balancer to drive more traffic to new idle server in case appearance of 503 Service unavailable. To address these problems, this paper provides a dynamic load balancing solution for kubernetes. The solution includes a load monitor, which can learn resource consumption and collect pod load, and a dynamic load balancing algorithm. The dynamic load balancing algorithm combines system load and service dependency to describe pod load, and finally select appropriate pod according to different resource consumption.

For future work, we will improve the scalability and reliability of ingress controller.  As we know, the ingress controller works as the proxy of services alone in kubernetes, once it goes wrong, the web services become unavailable. We plan to set up monitoring for ingress controller, if something goes wrong with it, a backup ingress controller will take place of it to work. If the workload is beyond the ability of current ingress controller, more ingress controllers should be started to spread the load, more and more, all these ingress controllers should present as the same user-transparent proxy. We hope this solution help ingress controller perform better in kubernetes.
